% DO NOT COMPILE THIS FILE DIRECTLY!
% This is included by the other .tex files.

\begin{frame}{What is ELF?}
    \begin{itemize}
        \item ELF stands for Executable and Linkable Format\footnote{\url{https://refspecs.linuxfoundation.org/elf/elf.pdf}}.
        \item It is a common standard file format for executables, object code, shared libraries, and core dumps.
        \item Originally developed by Unix System Laboratories and now widely used in Unix-like operating systems.
    \end{itemize}
\end{frame}

\begin{frame}{Structure of an ELF File}
    \begin{itemize}
        \item An ELF file consists of three main parts:
        \begin{itemize}
            \item \textbf{Header:} Contains metadata about the file type, architecture, and entry point.
            \item \textbf{Program Header Table:} Describes how the file should be loaded into memory.
            \item \textbf{Section Header Table:} Provides information about the sections in the file.
        \end{itemize}
        \item ELF files are designed to be flexible and extensible.
    \end{itemize}
\end{frame}

\begin{frame}{Benefits of ELF}
    \begin{itemize}
        \item Platform-independent format, enabling portability.
        \item Simplifies the linking and loading process.
        \item Supports dynamic linking, reducing redundancy.
        \item Extensively used in modern development environments.
    \end{itemize}
\end{frame}

\begin{frame}[fragile]
\frametitle{Binwalk Output}

\begin{lstlisting}[language=bash, basicstyle=\ttfamily, breaklines=true]
binwalk sample 
\end{lstlisting}

\begin{tabular}{ccp{0.6\textwidth}}
\textbf{Decimal} & \textbf{Hexadecimal} & \textbf{Description}\\
0                & 0x0                  & ELF, 32-bit LSB executable, Intel 80386, version 1 (SYSV) \\
13111            & 0x3337               & Boot section Start 0x58028941 End 0x5A41 \\
13115            & 0x333B               & Boot section Start 0x5A41 End 0x0 \\
\end{tabular}

\vspace{1cm}

$\to$ matched signatures
\end{frame}


\begin{frame}[fragile]
\frametitle{ELF Symbols from Binary Analysis}

Extract symbols from binary excluding GBLIBC references

\begin{lstlisting}[language=bash, basicstyle=\ttfamily, frame=single]
nm sample | grep -v GBLIBC
\end{lstlisting}


\begin{lstlisting}[basicstyle=\ttfamily, frame=single]
08048bfd t p4tch_sel1nux_codztegfaddczda
08048e9c t parse_cred
8050bb3 T prepare_fops_lsm_shellcode
08049215 t put_your_hands_up_hooker
0804b220 D r1ngrrrrrrr
0804988e t rey0y0code
0804b2c0 d ruujhdbgatrfe345
\end{lstlisting}
\end{frame}

\begin{frame}
\frametitle{ELF Symbols from Binary Analysis}

\begin{itemize}
    \item Interpretation of the output of tool \tt{nm}
    \item man page is your friend
\end{itemize}

\begin{tabular}{|c|p{0.7\textwidth}|}
\hline
\textbf{Symbol Type} & \textbf{Explanation} \\ \hline
a & The symbol's value is absolute and will not be changed by further linking. \\ \hline
b & The symbol is in the BSS data section. \\ \hline
d & The symbol is in the initialized data section. \\ \hline
r & The symbol is in the read-only data section. \\ \hline
t & The symbol is in the text (code) section. \\ \hline
w & The symbol is a weak symbol that has not been specifically tagged as a weak object symbol. \\ \hline
\end{tabular}

\end{frame}
