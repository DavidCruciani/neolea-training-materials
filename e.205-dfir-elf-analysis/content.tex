% DO NOT COMPILE THIS FILE DIRECTLY!
% This is included by the other .tex files.

\begin{frame}{What is ELF?}
    \begin{itemize}
        \item ELF stands for Executable and Linkable Format.
        \item It is a common standard file format for executables, object code, shared libraries, and core dumps.
        \item Originally developed by Unix System Laboratories and now widely used in Unix-like operating systems.
    \end{itemize}
\end{frame}

\begin{frame}{Structure of an ELF File}
    \begin{itemize}
        \item An ELF file consists of three main parts:
        \begin{itemize}
            \item \textbf{Header:} Contains metadata about the file type, architecture, and entry point.
            \item \textbf{Program Header Table:} Describes how the file should be loaded into memory.
            \item \textbf{Section Header Table:} Provides information about the sections in the file.
        \end{itemize}
        \item ELF files are designed to be flexible and extensible.
    \end{itemize}
\end{frame}

\begin{frame}{Benefits of ELF}
    \begin{itemize}
        \item Platform-independent format, enabling portability.
        \item Simplifies the linking and loading process.
        \item Supports dynamic linking, reducing redundancy.
        \item Extensively used in modern development environments.
    \end{itemize}
\end{frame}
